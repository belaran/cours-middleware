\ifslide{
  \section[Agenda]{}

  \setcounter{tocdepth}{2}

  \begin{frame}
    \begin{multicols}{2}
      \fontsize{8}{10}{
        \tableofcontents[part=1]

      }
    \end{multicols}
  \end{frame}
  \part{Introduction au Middleware}
}

\ifbook{
  \chapter{Le Middleware}
}

% \begin{frame}
%     \tiny{
%         \paragraph{
%     Le plan reste encore à raffiner mais la trajectoire de ce dernier consiste à partir de la
%     "position' du middleware au sein d'une architecture multi tiers - soit une vision macroscopique
%     du problème pour commencer, puis de faire un "zoom" sur les différents services ou frameworks
%     disponibles, pour en comprend l'utilité et le rôle. Pour chacun de ces derniers, on se
%     contentera de rester dans les "grandes lignes", l'objectif étant surtout de bien comprendre le
%     rôle du composant/service, et non son fonctionnement interne.}
% }
%
% \tiny{
%         \paragraph{La dernière partie se concentrera sur les problématiques de monté en charge et de
%         production. Là encore, on ne rentrera pas forcément dans les détails techniques mais poussé,
%         mais on illustrera les différents par des retours d'expérience - en mettant en exergue les
%         défaut de conceptions ou de mise en place qui ont été la cause des problèmes. L'objectif
%         pédagogiques avoués de cette partie, étant clairement de donner les "billes" aux futures
%         chef de projet pour savoir identifier en amont quand certains choix se font à la légère, ou
%         pouvoir remonter à leur hiérarchie l'importance de tel ou tel "détail technique"...}
%     }
% \end{frame}
\section{0 - Remise à niveau (09/01/2012)}

\section{A - Bases, terminologie et prérequis (09/01/2012)}
\begin{frame}
  \begin{block}{Focus}
    En introduction au cours aux suites de rappels et/ou définitions
    sont effectués dans ce chapitre pour assurer la mise en place d'une
    sémantique et d'une terminologie commune.    
  \end{block}
\end{frame}
\subsection{Processeur, entrées/sorties et réseaux}

\subsection{Qu'est ce que le middleware ?}

\subsection{Caractéristiques de la Programmation Orienté Objet}
\subsubsection{Héritage et Polymorphismes}
\subsubsection{Agrégation}
\subsubsection{Composant}

% \begin{frame}
%
%     \begin{block}{Points essentiels}
%         \begin{itemize}
%             \item rappel sommaire sur l'architecture 3 tiers, décrire que le middleware est en fait, tout ce qui se
% situe entre les requêtes (HTTP dans le cas du Web) et les accès aux base de données - en essence le
% middleware, c'est les briques qui composent l'application en tant que tel
%             \item premier survol de l'ensemble des besoins applicatifs (sécurité, gestion d'identité, cache,
% messaging...)
%             \item expliciter/décrire le besoin, dans la conception d'un IT, de découper son "business" en module,
% d'avoir est du code partagé (des Jars) mais aussi des applicatifs qui utilisent les mêmes briques,
% les mêmes services, etc...
%         \end{itemize}
%     \end{block}
% \end{frame}

\section{B - Les services applicatifs 1/2 (10/01/2012)}
\begin{frame}
  \begin{block}{Focus}
    Ciblé sur l'applicatif, cette partie du cours se concentre sur les
    services et composants proposés par les \textit{middlewares} à une
    application.
  \end{block}
\end{frame}
\subsection{Serveur d'applications}
\subsubsection{Pool(s) de connexion}
\subsubsection{Threads}

\subsection{Conteneur Web}
\subsubsection{Connecteur HTTP et Servlet}
\subsubsection{Session et Cookie}

\subsection{Transaction}

\subsection{Cache}

\section{B - Les services applicatifs 2/2 (10/01/2012)}
\subsection{BPM}

\subsection{Message}

\subsection{Sécurité}
% \begin{frame}
%   \begin{itemize}
%   \item Gestion d'identité (SSO, Annuaire,...)
%   \item Modèle de Sécurité (EJB, JAAS)
%   \end{itemize}
% \end{frame}
\subsection{Modèle de programmation (ex:EJB/Spring)}
% \begin{frame}
%   \begin{itemize}
%     \item gestion de l'état
%     \item concurrence
%   \end{itemize}
% \end{frame}

\section{C - Application distribué et intégration (11/01/2012)}
\begin{frame}
  \begin{block}{Focus}
    Elargissant le spectre au-delà du périmètre de l'application,
    cette partie étudie les services et composants proposés par les
    \textit{middlewares} pour dialoguer et s'intégrer à son environement, et
    plus spécialement à la nature \textbf{distribué} du système d'information
    dans lequel elle évolue.
  \end{block}
\end{frame}
\subsection{Source de données}

\subsubsection{Fichiers et resources}
\subsubsection{SQL et base de données relationnelle}
\subsubsection{Annuaire}
\subsubsection{NoSQL}
\subsection{Programmation distribué}
% \begin{block}{Historique}
%   \begin{itemize}
%     \item RPC
%     \item CORBA
%     \item Java-RMI
%     \item DCE
%     \item COM/DCOM
%   \end{itemize}
% \end{block}
\subsection{Transaction distribué}
\subsection{SOA}
\subsubsection{WebServices}
%   \begin{itemize}
%     \item SOAP et WSDL
%     \item ReST
%   \end{itemize}

\subsubsection{ESB}
\subsection{ETL, EAI et autres middlewares}

\section{D - Mise à échelle et production (12/01/2012)}
\begin{frame}
  \begin{block}{Focus}
    Cette partie évoque les problématiques liées à la mise en production 
    et la surveillance d'une application et des \textit{middlewares} qu'elle utilise.
  \end{block}
\end{frame}
\subsection{Problématique de performances des middlewares}

\subsection{Stratégie de mise à l'échelle (Ferme, clustering...)}

\subsection{Indicateur, surveillance et alertes}

