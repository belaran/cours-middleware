\mysection{C - Application distribuée et intégration (11/01/2012)}
\abstractframe{Élargissant le spectre au-delà du périmètre de l'application,
cette partie étudie les services et composants proposés par les
\textit{middlewares} pour dialoguer, et s'intégrer, à son environnement, et plus
spécialement à la nature \textbf{distribuée} du système d'information dans lequel
elle évolue.}{../img/overview-integration.png}

\subsection{Source de données}

\mysubsubsection{Fichiers et ressources}
\mysubsubsection{SQL et base de données relationnelle}
\mysubsubsection{Annuaire}
\mysubsubsection{NoSQL}
\mysubsection{Programmation distribuée}
% \begin{block}{Historique}
%   \begin{itemize}
%     \item RPC
%     \item CORBA
%     \item Java-RMI
%     \item DCE
%     \item COM/DCOM
%   \end{itemize}
% \end{block}
\mysubsection{Transaction distribuée}
\mysubsection{Architecture orientée service (SOA)}
\mysubsubsection{WebServices}
%   \begin{itemize}
%     \item SOAP et WSDL
%     \item ReST
%   \end{itemize}

\mysubsubsection{Bus logiciel (ESB)}
\mysubsection{ETL, EAI et autres middlewares}

\mysection{D - Mise à échelle et production (12/01/2012)}

\mysubsection{Problématique de performances des middlewares}

\mysubsection{Stratégie de mise à l'échelle (Ferme, clustering...)}

\mysubsection{Indicateur, surveillance et alertes}

\abstractframe{Cette partie évoque les problématiques liées à la mise en production
et la surveillance d'une application et des \textit{middlewares} qu'elle
utilise.}{../img/overview-monitoring.png}
