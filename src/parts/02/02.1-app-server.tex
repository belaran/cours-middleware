\subsection{Serveur d'applications}

\ifbook{
  % disclaimer
\paragraph{} Le champ d'expertise de l'auteur de ce document étant concentré sur le monde Java, et
plus spécifiquement , le monde des produits JBoss, et cet état de fait n'est pas sans conséquence
sur ce cours. En effet, bien que la plupart des concepts évoqués dans cette partie se retrouvent
dans d'autres univers technologique, certains sont néanmoins parfois relativement spécifique à Java
ou JBoss.

\paragraph{} Dans la mesure du possible, lorsque ces cas seront évoqués, leur spécifités sera
soulignés. Néanmoins, le lecteur devra rester vigilant, car il n'est pas toujours facile de faire se
distingo.


- What is an app server ? définition, penser aussi à Rails ou WebWare par exemple
- Produit d'intégration, offre API standard, gère les prob. d'intg à la place de l'application
- externalise configuration,


\subsubsection{Pool(s) de connexion}

% TODO: definition d'un pool de connexion, intérêt...

\paragraph{} Plutôt que de continuer de faire un résumé très théorique, et vraisemblablement peu
pertinent sur ce sujet - qui ne ferait que satisfaire un goût très Mésopotamien de l'inventaire et
des catalogues, nous allons voir un petit cas pratiques de configuration d'un pool de connexion.


\paragraph{} Cette démarche sera, espérons-le, un peu moins aride, et devrait surtout permettre au
lecteur de bien saisir les concepts sous jacents. Pour être didactique, cette section contiendra
donc des extraits de codes et de configuration, mais là encore, l'objectif pédagogique ne sera pas
retenir, ni même de forcément comprend l'intégralité de ces extraits, mais de bien cerner, de
manière concrète, les concepts de plus haut niveau qui s'y rapportent.

% TODO: RMI appli avec une connecion, qui dure 60s, renvoie une * toutes les 5s - barre de
% progression - timeout, puis 2 utilisateurs concurrent

- HTTP, RMI, AJP => offre pool de connexion, partagé entre les applications



\subsubsection{Threads}
