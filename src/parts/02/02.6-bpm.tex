\mysubsection{Gestion des processus métiers (BRMS)}

\ifbook{
  \paragraph{} Une des problématiques les plus récurrentes dans les projets informatiques est la
  difficulté, pour les développeurs, de comprendre les besoins fonctionnelles associées à
  l'application qu'ils développent. Un lourd de spécification est généralement réalisé alors pour
  permettre aux développeurs de bien comprendre les différents aspects, très proche du métier, que
  doit réaliser le logiciel.

  \paragraph{} En outre, la nature changeante de la plupart des professions fait qu'il est rapidement
  nécessaire de modifier une partie de la logique des logiciels qui leur sont associé. Là encore, un
  lours travail de transfert est à effectuer entre les personnes disposant des réels connaissances
  métiers et les développeurs.

  \paragraph{} On notera d'ailleurs que cette problématique explique, pour beaucoup, le succès de
  feuilles de calculs (qu'il s'agisse de \mylink{}{Excel} ou \mylink{}{Libre Office Calc}). En fait,
  ces derniers permettent à des comptables ou des fiscalistes, par exemple, de facilement se fabrique
  une sorte de "petit programme" faisant exactement ce dont ils ont besoin, sans passer par
  l'intermédiaire d'un dévelopeur. On trouve d'ailleurs de nombreux expert de ce genre d'outil dans
  les salles de marché, où il est souvent nécessaire de concevoir un logiciel de calcul avancé en
  quelques minutes seulement...

  \paragraph{} Le constat que venons d'effectuer à amener une nouvelle technologie, les
  \textbf{moteurs de règle} à émerger - ou en anglais \textit{Business rule management system}. L'idée
  sous jacente est, comme souvent, assez simple: fournir un outil générique conçu pour implémenter et
  concevoir des règles dites "métiers". Les règles définis à l'aide de cet outil s'intègrera donc ensuite
  dans une application plus vaste.

  \paragraph{} Ainsi, des experts fonctionnels, avec peu ou pas de compétence de programmation,
  pourront être en charge de l'implémentation des règles métiers, et les développeurs pourront
  aisément intégrer leur travail au sein de l'applicaion. Les avantages de cette approche sont
  flagrants:

  \begin{itemize}
    \item diminuer le besoin de développement pour modifier les fonctionnalités purement métier d'une
    application
    \item donner plus de contrôle sur la logique de décision et la gestion du processus métier en
    général à des experts fonctionnels plutôt qu'à des développeurs.
  \end{itemize}

  \paragraph{} Le domaine étant encore finalement assez récent, on ne trouve pas encore
  \textbf{standard} - bien que des initiatives dans ce sens avance rapidement\footnote{Comme par
  exemple la standardisation d'une API Java avec la JSR-94}, et il existe donc
  encore de grande différences entre les produits \textit{middleware} disponibles sur le marché. On
  peut néamoins retenir que tout \textbf{moteur de règles} est constitué des éléments suivants:

  \begin{description}
      \item[dépôt] ce dernier permet de aisément externaliser le code "métier" du reste de
      l'application.
      \item[outils] permettant à la fois aux développeurs et aux spécialistes métiers de travailler
      ensemble.
      \item[conteneur d'exécution\footnote{Oui, encore un nouveau conteneur d'exécution...}] pour
      exécuter les règles définies pour l'application et interagir avec l'application.
  \end{description}

  \paragraph{} Un exemple relativement complet, et ouvert, de ce type de \textit{middleware} est sans
  conteste \mylink{https://community.jboss.org/wiki/JBossRules}{JBoss Rules}, anciennement appelé
  Drools.

}

\ifslide{

\begin{frame}
  \begin{block}{Disclaimer}
    \begin{center}
      \Large{I suck}
    \end{center}
  \end{block}
\end{frame}


\begin{frame}{Business Process Management}
  \begin{block}{Motivations}
    \begin{itemize}
      \item developer/business expert
      \item lifecycle
    \end{itemize}
  \end{block}
\end{frame}


\begin{frame}
  \begin{block}{What makes a BRM ?}
    \begin{itemize}
      \item \textbf{Repository} %, allowing decision logic to be externalized from core application code
      \item \textbf{Tools} % allowing both technical developers and business experts to define and manage decision logic
      \item \textbf{Runtime environment} %, allowing applications to invoke decision logic managed within the BRMS and execute it using a business rules engine
     \end{itemize}
  \end{block}
\end{frame}


%The top benefits of a BRMS include:[2]
\begin{frame}
  \begin{block}{What's the point ?}
    \begin{itemize}
      \item Reduced or removed reliance on IT departments for changes in live systems
      \item Increased control over implemented decision logic for compliance and better business management
      \item The ability to express decision logic with increased precision %, using a business vocabulary syntax and graphical rule representations (decision tables, trees, scorecards and flows)
      \item Improved efficiency of processes through increased decision automation
      \item Most BRMS vendors have evolved from rule engine vendors to provide business-usable software development lifecycle solutions, based on declarative definitions.
    \end{itemize}
  \end{block}
\end{frame}

\begin{frame}
  \begin{block}{Standardisation}
    \begin{itemize}
      \item No standard for business rules defined within a BRMS
      \item a standard for a Java Runtime API for rule engines JSR-94.
    \end{itemize}
  \end{block}

  \begin{block}{Other standards (under development)}
    \begin{itemize}
      \item OMG Business Motivation Model (BMM): A model of how strategies, processes, rules, etc fit together for business modeling
      \item OMG SBVR: Targets business constraints as opposed to automating business behavior
      \item OMG Production Rule Representation (PRR): Represents rules for production rule systems that make up most BRMS' execution targets
      \item W3C RIF: A family of related rule languages for rule interchange
      \item Many standards, such as domain-specific languages, define their own representation of rules, requiring translations to generic rule engines or their own custom engines.
    \end{itemize}
  \end{block}
\end{frame}
}
