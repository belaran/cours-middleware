\ifbook{
  \subsubsection{Le Web}
  \paragraph{} Avec l'arrivé des technologies du "Web", soit essentiellement au début, HTML et HTTP,
  le mouvement de balancier évoqué une précédente section s'inverse de nouveau. Dans leurs
  laboratoires, les concepteurs initiale de ce protocole de communication visèrent avant tout la
  simplicitié pour assurer surtout une bonne compatibilité entre les différents systèmes existants
  de part le monde.

  \paragraph{} En effet, avant d'aller plus loin, il est important de se rappeler que à cette
  époque, il existait de nombreux systèmes d'exploitation, tous différents et rarement interopérable
  entre eux. Les plus connues aujourd'hui Apple et Windows, mais à l'époque s'ajouté aussi de
  nombreux types d'Unix différents (cf. \textit{Unix Wars} [TODO]), accompagné par nombreux autres
  systèmes, tel que OpenVMS.

  \paragraph{} Sans compter que ces systèmes avaient même des protocoles de communications
  différents. Pour reprendre l'exemple de OpenVMS, ce système utilisé un protocole propriétaire à
  son constructeur, intitulé DecNet et non le standard \textit{de facto} d'aujourd'hui, le protocole
  TCP/IP.

  \paragraph{} Ainsi, concevoir un système applicatif portable - c'est à dire qui fonctionnerait à
  peu près partout et permettrait de communiquer aisément entre des systèmes différents n'était pas
  une simple tâche. C'est donc pour ces raisons, en autres, que les concepteurs de HTTP et de HTML
  ont choisi de faire très simple.

  \paragraph{} Le protocole en lui même est un simple protocole \textbf{texte}, et non binaire, il
  est donc aisé de l'implémenter et, au besoin, de regarder l'échange en lui même pour comprendre la
  source d'un problème. Comme les données échangés sont du texte, tout systèmes de l'époque, aussi
  différent soit-il des autres, était capable de le comprendre.

  %TODO: Demo console de HTTP

  \paragraph{} Toujours par souci de simplicité, l'objectif était dans les faits assez peu ambitieux,
  puisqu'il s'agissait d'afficher du contenu simple, du texte un peu enrichi, pour permettre en fait
  au milieu universitaire de publier, facilement et rapidement,  à l'intention de leurs confrères,
  des informations.

  \paragraph{} La conséquence directe de ce choix simple de données utilisés à été de limiter le
  rôle du client à afficher, du mieux possible selon le systèmes utilisées, les informations
  fournies par le serveur HTTP. Le \textbf{client lourd} venait de faire un régime et redevenait un
  \textbf{client léger}.

  % TODO: Dessins de synthèse
}


\ifslide {
  \begin{frame}{Le "Web"}
  - evolution vers Web
  \end{frame}
}

\ifbook{
  \subsubsection{Les technologies "Web" s'enrichissent}
  % Formulaire et session HTTP
  \paragraph{} Le protocole HTTP, et son format de donnée, HTML, a eu le succès qu'on connait, et
  rapidement, malgré l'élégance de la simplicité de la solution, il apparu clair que d'\textbf{absence
  d'interfaction} entre l'utilisateur et le serveur HTTP était une limite trop contraignante. En
  effet, tel que nous l'avons évoqué jusqu'à maintenant, le modèle ne permet en essence que de
  télécharger un page au contenu \textbf{statique}.

  \paragraph{} Pour introduire plus d'interactivité, et permettre au serveur HTTP de modifier
  \textbf{dynamiquement} le contenu des pages présentées selon les demandes des utilisateurs, les
  formulaires ont été introduit. Ces derniers, associés à la méthode HTTP POST, ont donc permis aux
  utilisateurs d'envoyer des données au serveur HTTP, qui furent le point de départ des premières
  applications "web".

  \paragraph{} Mais une fois qu'il fût possible d'envoyer ces données, d'autres limites firent leur
  irruptions. La plupart des applications nécessitant souvent plusieurs échanges entre l'utilisateur
  et le serveur, il fût nécessaire d'ajouter un mécanisme de son coté pour lui permettre de
  conserver les données associées à l'utilisateur, ou plutôt à sa \textbf{session}.

%  \paragraph{} L'ajout donc de la notion de session HTTP permis là encore de contourner les limites
%  du modèle original, mais elle ne suffit pas entièrement. De la même manière dont le serveur avait
%  besoin de garder trace de son utilisateur, il fallait aussi que l'on soit en mesure, coté client,
%  de conserver une
  % CSS et Javascript
  \paragraph{} Alors que la première génération de site web finissaient de fleurir, une
  problématique, jusque là invisible, apparu de plus en plus clairement. Le HTML, dans toute la
  beauté de sa simplicité, enfreint par son essence même, une règle pourtant fondamentale de
  l'informatique, que nous avons déjà évoqué avec les bases données : la séparation du contenu et de
  sa présentation.

% TODO: Extrait de fichier HTML

  \paragraph{} En effet, au sein d'une page HTML, on mélange avec allégresse texte avec sa
  présentation qu'il s'agisse de le mettre en gras ou ialic, ou bien de le positionner au sein de la
  page. Et cet état fait rendu rapidement très difficile de faire évoluer, du point fonctionnel, les sites -
  puisque les graphistes ou ergonomes ne pouvait pas travailler de manière indépendantes des
  programmeurs, mais aussi de faire évoluer leurs chartes graphiques, puisque leurs simples mise à
  jour nécessitant de modifier l'intégralité des pages...

  \paragraph{} C'est ainsi que est apparu le CSS, \textit{Cascading Style Sheet}, ou plus
  simplement, les feuilles de styles, dont l'objectif était non seulement de permettre de modifier
  la présentation d'une page HTML existante, mais aussi de séparer la partie présentation d'un site,
  de ses données.

  \paragraph{} En parallèle à l'utilisation du CSS, un autre langage a été introduit, sous forme de
  script, placés dans la page, mais dont l'exécution est déclencheur par le navigareur et qui donc
  fonctionne non plus le serveur, mais sur l poste client.

  \paragraph{} Introduit tout d'abord de manière propriétaire, ce petit langage permis de rendre les
  sites "web" tellement plus dynamique et \textbf{conviviale}, en améliorant grandement
  l'interfaction de l'utilisateur avec ces dernirs, qu'il fût rapidement adopté, et même normalisé
  sous le nom, rarement utilisé, de ECMAScript.
}

% TODO: JVM ?
% TODO: Architecture 3 tiers en guise de conclusion

\ifbook{
  \subsubsection{Architecture n-tiers}
    \paragraph{} Si nous résumons les différents éléments évoqués jusqu'à maintenant sous forme de
    la figure ci dessous (TODO), on constate que désormais notre application, qui au départ
    s'exécutait sur une seule machine, relié par un terminal très primaire, est désormais réellement
    éclatée entre plusieurs composant, ou \textbf{tiers}.
    %TODO: dessin
}

\subsection{Qu'est ce que le middleware ?}

\ifbook {
  \paragraph{} Après ce vaste état des lieux, nous allons enfin pouvoir rentrer dans le thème de ce
  cour: le \textit{Middleware}. En premier lieu, essayons de trouver une définition un plus parlante
  de ce terme, qui n'est pas réellement de traduction française.

  \paragraph{} Si on on traduit très littéralement ce terme, on obtient quelques chose de l'ordre du
  "matériel du milieu". Bon, c'est peu parlant, mais clairement le suffix "\textit{-ware}" fait echo
  aux termes \textit{"software"} - le matériel logiciel, et \textit{"hardware"}, le matériel
  physique, ce qui signifie que, en fait, le mot clé ici, est le "milieu".

  \paragraph{} Mais de quoi exactement sommes nous au milieu ici ? Revoyons simplement notre dessin
  d'architecture à 3 tiers:

  %TODO: Dessin

  \paragraph{} Le \textit{Middleware} est tout simplement ce qui se retrouve, littéralement, au
  milieu ! Soit entre la base de données et les clients utilisées par les usagers du système. Mais
  au milieu, il y a l'application, non ? Certes oui, mais, celle ci s'appuye désormais plus sur les
  seules API fournies par le système d'exploitation, mais sur une kyrielle de services, soient fourni
  par son environnement d'exécution - tel que une machine virtuelle comme celle de Java ou C\#, ou un
  serveur d'application, ou par des composants qu'elle embarque.

  \paragraph{} Mais qu'apportent ces services de plus ? Nous allons étudier le "catalogue" en
  détails durant ce cours, mais en quelques mots, ils apportent beaucoup ! Gestion de \textit{pool}
  de connexion, cache locale ou distribué, sécurité, intégration avec un service d'authentification
  distant ou même encore la capacité de mettre en \textit{cluster} l'application pour assurer
  aisément sa montée en charge simplement en ajoutant de nouvelles machines.

  \paragraph{} Pour en revenir à une sorte de définition du \textit{Middleware}, nous retiendrons
  donc que ce dernier est l'ensemble des services utilisées par l'application, aussi bien en son
  sein, que pour communiquer avec des services distants. Bref, en essence, le \textit{Middleware} se
  trouve bel et bien "au milieu", entre l'application et le reste du monde...}

%- dessin "application" => "pile, gestion de threads, cache, ...", puis vers le modèle java/jee de container
%- évoqué la problématique du packaging et de la réutilisation de code, probablématique de gestion de version aussi
