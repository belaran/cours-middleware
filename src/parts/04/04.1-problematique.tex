\mysection{D - Mise à échelle et production (12/01/2012)}

\abstractframe{Cette partie évoque les problématiques liées à la mise en production
et la surveillance d'une application et des \textit{middlewares} qu'elle
utilise.}{../img/overview-monitoring.png}

\mysubsection{Problématique de performances des middlewares}

\ifslide{
  \begin{frame}
    \begin{block}{Performance et mise production des \textit{middlewares}}
      \begin{itemize}
        \item au coeur du problème
        \item complexité
        \item charge élévée
        \item attente accrue
        \item plus de période de repos
      \end{itemize}
    \end{block}

    \begin{block}{Effet "boite noire"}
      \begin{itemize}
        \item l'application en elle même
        \item rôle et fonctionnement des \textit{middleware}
        \item complexité et distribution
      \end{itemize}
    \end{block}
  \end{frame}
}


\mysubsection{Stratégie de mise à l'échelle (Ferme, clustering...)}

\ifslide{
  \begin{frame}
    \begin{block}{Objectifs}
      \begin{itemize}
        \item Tolérance aux pannes (et reprise sur pannes)
        \item Haute Disponibiités
      \end{itemize}
    \end{block}
    \begin{block}{Techniques}
      \begin{itemize}
        \item Balance de charge
        \item Ferme de serveurs
        \item \textit{Cluster}
      \end{itemize}
    \end{block}
  \end{frame}
}

\mysubsection{Indicateur, surveillance et alertes}

\ifslide{
  \begin{frame}
    \begin{block}{Problématique de la surveillance}
      \begin{itemize}
        \item Que faut il surveiller ? Que faut il mesurer ?
        \item Quelles alertes ?
      \end{itemize}
    \end{block}
  \end{frame}

  \begin{frame}
    \begin{center}
    \Large{Le \textit{monitoring} applicatif est forcément spécifique !}
    \end{center}

    \begin{block}{Définir sa stratégie}
      \begin{itemize}
        \item Maitriser la complexité de ses applicatifs
        \item Choisir ses indicateurs
        \item Donner les moyens de surveiller son applicatifs
      \end{itemize}
    \end{block}
  \end{frame}

  \begin{frame}{\textit{Time to draw some clever shit on the board...}}
  \end{frame}

}
