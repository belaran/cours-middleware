% Definition de l'affichage de l'ensemble des codes
\definecolor{white}{rgb}{1,1,1}
\definecolor{monbleu}{rgb}{0.03, 0.45, 0}
\lstset{
	language=Java,
	backgroundcolor=\color{white},
	basicstyle=\small,
	commentstyle=\color{monbleu},
	keywordstyle=\color{blue}\bfseries\emph
}

\newcommand{\rpeframe}[2]{
	\ifthenelse{ \boolean{slides} }
	{
		\begin{frame}
			\frametitle{#1}
			#2
		\end{frame}
	}
	{
		% TODO: #1 !
		#2
	}
}

\newcommand{\ifslide}[1]{
	\ifthenelse{ \boolean{slides} }
	{#1}
	{}
}

\newcommand{\ifbook}[1]{
	\ifthenelse{ \boolean{slides} }
	{}
	{#1}
}

\newcommand{\mylink}[2]{\href{#1}{\textcolor{blue}{#2}}}
\newcommand{\myphotocredit}[2]{\newline\tiny{\href{#1}{\textcolor{gris-doux}{#2}}}}

\newcommand{\abstractframe}[2]{
\begin{frame}
  \ifthenelse{ \boolean{slides} }{
    \begin{block}{Résumé du thème}
      \tiny{#1}
    \end{block} }{
    \begin{Sbox}
      \begin{minipage}{6in}
        \begin{center}\textbf{Résumé du chapitre}\end{center}
        \paragraph{}\textit{#1}
      \end{minipage}
      \end{Sbox}
      \fbox{\TheSbox}
  }
  \begin{center}
    \ifthenelse{ \boolean{slides} }
    {\includegraphics[width=220px]{#2}}{
      \begin{figure}[hb]
        \begin{center}
          \includegraphics[width=\textwidth]{#2}
          \caption{Focus du chaptire}
        \end{center}
      \end{figure}
      \newpage
    }
  \end{center}
\end{frame}
}

\newcommand{\demoframe}[2]{
  \begin{frame}{#1}
    \begin{center}
    Et maintenant une rapidement démonstration...
    \end{center}
  \end{frame}

}
